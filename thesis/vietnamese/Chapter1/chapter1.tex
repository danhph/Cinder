%*******************************************************************************
%*********************************** First Chapter *****************************
%*******************************************************************************

\chapter{MỞ ĐẦU}
\label{chap:problem-statement}
\graphicspath{{Chapter1/Figs/}}

\begin{chapabstract}
{Chương \ref{chap:problem-statement} đặt vấn đề cho đề tài, bao gồm sự nguy hiểm của phần mềm độc hại, tổng quan về tình hình nghiên cứu, ứng dụng các phương pháp máy học vào nhận diện mã độc, và đặc biệt là trình bày những hạn chế của các phương pháp đã có, cũng chính là lí do của việc chọn đề tài.}
\end{chapabstract}
Malware thường được sử dụng như một thuật ngữ chung để chỉ bất kỳ phần mềm nào được thiết kế nhằm mục đích gây thiệt hại cho một máy tính, máy chủ hoặc mạng máy tính \cite{moir2003defining}. 
Một sự cố từ malware có thể gây thiệt hại hàng triệu đô la, cụ thể, mã độc zero-day ransomware WannaCry đã gây ra thảm họa trên toàn thế giới từ việc đánh sập hệ thống của các Bệnh viện Dịch vụ Y tế Quốc gia Vương quốc Anh, cho đến việc tắt toàn bộ hệ thống sản xuất của công ty Honda tại Nhật Bản \cite{chen2017automated}.
Hơn nữa, phần mềm độc hại ngày càng trở nên tinh vi và đa dạng hơn mỗi ngày \cite{shahi2009technology}. Theo đó, bài toán phát hiện phần mềm độc hại là một vấn đề quan trọng trong an ninh mạng, đặc biệt khi xã hội trở nên phụ thuộc nhiều vào các hệ thống máy tính.

Các sản phẩm nhận diện malware trước đó thường sử dụng các phương pháp dựa trên quy tắc (rule-based) hoặc dựa trên chữ ký (signature-based), yêu cầu các nhà phân tích xử lý các quy tắc thủ công (handcraft rules) có liên quan để phát hiện mã độc.
Cách tiếp cận này có độ chính xác cao.
Tuy nhiên, các quy tắc này  thường là cụ thể và không thể nhận ra phần mềm độc hại mới, ngay cả khi nó sử dụng cùng chức năng.
Vì lý do này, ý tưởng phát hiện mã độc dựa trên các thuật toán học máy được phát sinh.
Thuật toán học máy học các mẫu cơ bản (pattern) từ một tập huấn luyện nhất định, bao gồm cả các mẫu độc hại và lành tính.
Những pattern phân biệt các phần mềm độc hại từ phần mềm lành tính.
Kể từ khi Schultz và các cộng sự chó thấy sự hiệu quả của việc ứng dụng các thuật toán máy học vào nhận diện mã độc \cite{schultz2001data}, máy học đang trở thành một trong những công cụ phổ biến và có nhiều ảnh hưởng trong việc đảm bảo an toàn cho hệ thống.

Một số phương pháp sử dụng học máy đã mang lại các mô hình quá tích cực, thể hiện độ chính xác dự báo đáng kể, nhưng đã dẫn đến nhiều những kết quả dương tính giả (false positives). Các kết quả dương tính giả làm tiêu cực trải nghiệm của người dùng, ngăn không cho triển khai các phần mềm mới. Theo khảo sát của các IT administrator năm 2017 \cite{jonathan2017survey}, 42\% các công ty cho rằng người dùng của họ bị mất năng suất là do liên quan đến những kết quả dương tính giả, tạo ra một điểm nghẹt cho các quản trị viên CNTT trong môi trường doanh nghiệp. Các kỹ sư bảo mật cũng thông báo các báo động giả này thường gây rối khi họ đang làm việc để phát hiện và loại bỏ phần mềm độc hại.
Một báo cáo được công bố vào năm 2015 cũng cho thấy rằng nhiều tổ chức ở Hoa Kỳ đã tiêu thụ một lượng tiền khổng lồ để xử lý các cảnh báo phần mềm độc hại không chính xác \cite{eduard2015false}. Do đó, ngay cả khi một giải pháp có tỷ lệ phát hiện cao nhất, nếu nó có một số lượng lớn các kết quả dương tính giả, nó được xem là vô dụng hơn so với một giải pháp với các kết quả dương tính giả thấp và một tỷ lệ phát hiện vừa phải.

Chúng tôi chọn đề tài "Phát hiện Mã độc nằng phương pháp Máy học trên hệ điều hành Windows" với mong muốn đóng góp một phương pháp mới để giải quyết vấn đề xác định phần mềm độc hại, đạt được tỷ lệ phát hiện cao và tỷ lệ dương giả thấp.
