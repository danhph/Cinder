%*******************************************************************************
%****************************** Second Chapter *********************************
%*******************************************************************************

\chapter{Giới Thiệu Tổng Quan}
\label{chap:introduction}
\graphicspath{{Chapter2/Figs/}}

\begin{chapabstract}
Chương \ref{chap:introduction} trình bày tổng quan về việc áp dụng Học máy trong Phát hiện mã độc tĩnh trên các hệ điều hành Windows; động lực và mục tiêu của nghiên cứu này; liệt kê các nghiên cứu có liên quan của tác giả ở Việt Nam hoặc trên toàn thế giới.
\end{chapabstract}

\section{Tổng quan}
\label{sec:overview}

Phần mềm độc hại là phần mềm được thiết kế để xâm nhập hoặc gây hại cho hệ thống máy tính mà không có sự đồng ý của chủ sở hữu.
Các phân loại đơn giản phần mềm độc hại là nhận dạng tệp phần mềm độc hại và tệp sạch.
Phát hiện phần mềm độc hại tĩnh đang phân loại các mẫu là độc hại hoặc lành tính mà không cần thực thi chúng.
Ngược lại, phát hiện phần mềm độc hại động phát hiện phần mềm độc hại dựa trên hành vi thời gian chạy của phần mềm độc hại \cite{athiwaratkun2017malware, dahl2013large}. 
Mặc dù phát hiện phần mềm độc hại tĩnh được biết đến là không thể giải quyết được toàn bộ bài toán nhận dạng mã độc \cite{cohen1987computer}, nó là một lớp quan trọng trong một bộ bảo mật bởi vì khi thành công, nó cho phép xác định các tập tin độc hại trước khi thực thi.

Bên cạnh đó, máy học là một công cụ hấp dẫn cho một khả năng phát hiện các mẫu mới và khả năng phát hiện heuristic dựa trên mẫu đã có (phần \ref{ssec:machine-learning-intro}).
Các mô hình học tập được giám sát (Supervised learning models) sẽ tự động tìm ra mối quan hệ phức tạp giữa các thuộc tính tệp trong dữ liệu huấn luyện và phân biệt giữa các mẫu độc hại và lành tính (phần \ref{ssec:supervised-learning}).
Hơn nữa, các mô hình học máy sẽ khái quát hóa với tập dữ liệu mới có các tính năng và nhãn theo một mẫu tương tự với dữ liệu đào tạo.

Ngoài ra, trong các hệ điều hành Windows, định dạng chung cho phần mềm độc hại là định dạng Portable Executable (PE) (phần \ref{sec:pe-file}), đó là định dạng tệp cho các tệp thi hành, mã đối tượng, tệp DLL, tệp phông chữ FON và các tệp khác được sử dụng trong cả phiên bản 32 bit và 64 bit. Định dạng PE đóng gói thông tin cần thiết cho trình tải hệ điều hành Windows để quản lý mã thực thi được bao bọc.

Do đó, nhiều phương pháp phát hiện phần mềm độc hại PE dựa vào các phương pháp học máy đã được đề xuất bắt đầu từ năm 1995 \cite{kephart1995biologically,schultz2001data,kolter2006learning,Shafiq2009AFF,saxe2015deep}.
Năm 2001, Schultz và cộng sự biểu diễn các tệp PE theo các tính năng bao gồm imported functions, strings và byte sequences \cite{schultz2001data}. 
Các mô hình bao gồm các quy tắc được tạo ra từ RIPPER \cite{cohen1995fast}, Naive Bayes và một ensemble classifier.
Phương pháp này đã được mở rộng bởi Kolter et al. vào năm 2006 \cite{kolter2006learning} by cách sử dụng byte-level N-grams và những kỹ thuật từ xử lý ngôn ngữ tự nhiên, bao gồm trọng số TFIDF cho các strings. 
Vào năm 2009, Shafiq và các cộng sự đề xuất chỉ sử dụng bảy tính năng từ PE header, do thực tế là hầu hết các ứng dụng phần mềm độc hại trong nghiên cứu của họ thường trình bày các yếu tố đó \cite{Shafiq2009AFF}. 
Năm 2015, Saxe và Berlin sử dụng two-dimensional byte entropy histograms và một multi-layer neural network cho việc nhận dạng mã độc \cite{saxe2015deep}.

\section{Động lực}
\label{sec:motivation}

Mặc dù nhiều mô hình đã thực hiện độ chính xác dự báo nổi bật, chúng được huấn luyện và xác thực trên tập dữ liệu mất cân bằng vì bài toán phát hiện phần mềm độc hại không nhận được sự chú ý tương tự như các bài toán khác trong cộng đồng nghiên cứu mở. 
Những hạn chế pháp lý là thách thức chính cho việc công bố tập dữ liệu điểm chuẩn để phát hiện phần mềm độc hại, cụ thể, Saxe và Berlin không thể công bố dữ liệu hoặc mã nguồn cho dự án của họ do tính chất pháp lý và độc quyền của nó \cite{saxe2015deep}. 
Ngoài ra, không giống như hình ảnh, văn bản và lời nói có thể được gắn nhãn gần như ngay lập tức và trong nhiều trường hợp không cần chuyên gia, việc phân loại tệp độc hại hay lành tính là quá trình tốn thời gian cho cả những chuyên viên được đào tạo tốt.
Công việc ghi nhãn có thể được tự động thông qua phần mềm anti-malware, nhưng kết quả có thể là độc quyền hoặc được bảo vệ pháp lý khác. 
Do đó, các mô hình này có thể không gây ấn tượng với dữ liệu cân bằng \cite{chawla2009data}.

Hơn nữa, một phương pháp hiệu quả, có độ chính xác ấn tượng và tỷ lệ dương tính giả rất thấp, sẽ ngăn chặn những tổn thất lớn từ phần mềm độc hại, cũng như mang lại trải nghiệm tốt cho người dùng và tiết kiệm tài nguyên cho nhiều tổ chức.

\section{Mục tiêu}
\label{sec:objectives}

Như đã đề cập trong phần \ref{sec:motivation}, kết quả từ nhiều phương pháp được đề xuất có thể không ấn tượng khi đánh giá với dữ liệu cân bằng.
Vì vậy mục tiêu chính của luận án này là \textbf{áp dụng các phương pháp học máy trong phát hiện phần mềm độc hại với tập dữ liệu cân bằng}. 
Ngoài ra, mục tiêu dự kiến là đề xuất một phương pháp với \textbf{tỷ lệ phát hiện cao} và \textbf{tỷ lệ báo động giả rất thấp}. 
Cuối cùng, chúng tôi hướng đến kiến thức về học máy và cách áp dụng chúng để tăng cường bảo mật thông tin người dùng trong việc giải quyết vấn đề phát hiện phần mềm độc hại.

Công việc chi tiết mà chúng tôi đã thực hiện trong luận án này bao gồm:

\begin{itemize}
    \item Nghiên cứu và xây dựng nền tảng về bảo mật thông tin, đặc biệt là phần mềm độc hại.
    \item Nghiên cứu và hiểu các kiến thức cơ bản trong Học máy bao gồm phân loại, thuật toán cây quyết định, thuật toán Random Forest, thuật toán Support Vector Machine, Neural Networks và Gradient-Boosting Decision Tree.
    \item Nghiên cứu và hiểu rõ hơn về thuật toán cây Gradient-Boosting Decision Tree.
    \item Áp dụng thuật toán Gradient-Boosting Decision Tree vào bài toán phát hiện phần mềm độc hại. Tiến hành các thí nghiệm để đánh giá hiệu suất của mô hình và tối ưu hóa các tham số.
\end{itemize}

\section{Những nghiên cứu liên quan}

Phát hiện phần mềm độc hại đã phát triển trong vài năm qua, do mối đe dọa ngày càng tăng gây ra bởi phần mềm độc hại cho các doanh nghiệp lớn và các cơ quan chính phủ.
Tại Việt Nam, PGS.TS. Vũ Thanh Nguyên và cộng sự đề xuất một phương pháp kết hợp của thuật toán lựa chọn tiêu cực và mạng miễn dịch nhân tạo để phát hiện virus \cite{nguyen2014combination}, và phương pháp phát hiện phần mềm độc hại biến chất bằng phân tích Portable Executable (PE) cùng Longest Common Sequence (LCS) \cite{vu2017metamorphic}.
Nguyen Van Nhuong và các cộng sự đã đề xuất phương pháp semantic để phát hiện phần mềm độc hại biến chất một cách hiệu quả \cite{van2014semantic}.

Các phương pháp phát hiện phần mềm độc hại có thể được phân loại trong phát hiện phần mềm độc hại tĩnh hoặc phát hiện phần mềm độc hại động \cite{egele2012survey}.
Về lý thuyết, tính năng phát hiện động cung cấp chế độ xem trực tiếp về hành động của phần mềm độc hại, ít bị ảnh hưởng hơn bởi việc làm xáo trộn mã thực thi và làm cho việc sử dụng lại phần mềm độc hại trở nên khó khăn hơn \cite{moser2007limits}.
Tuy nhiên, trên thực tế, phần mềm độc hại có thể xác định liệu nó có đang chạy trong một hộp cát, và ngăn chính nó thực hiện hành vi nguy hiểm \cite{vidas2014evading}.
Điều này dẫn đến một cuộc chạy đua vũ trang giữa các phương pháp phát hiện phần mềm độc hại động và phần mềm độc hại.
Hơn nữa, trong nhiều trường hợp, phần mềm độc hại không hoạt động một cách chính xác do thiếu dependency hoặc cấu hình hệ thống không mong muốn.
Những vấn đề này gây khó khăn cho việc thu thập tập dữ liệu về hành vi phần mềm độc hại.

Ngược lại, phân tích tĩnh không yêu cầu các thiết lập phức tạp, tốn kém để thu thập, nó có các tập dữ liệu khổng lồ có thể được tạo bằng cách tổng hợp các tệp nhị phân.
Điều này làm cho phát hiện phần mềm độc hại tĩnh rất phổ biến với phương pháp học máy, có xu hướng hoạt động tốt hơn khi tăng kích thước dữ liệu \cite{banko2001scaling}.
Một số chương trình phát hiện phần mềm độc hại dựa trên học máy đã được giới thiệu từ ít nhất năm 1995 \cite{kephart1995biologically,schultz2001data,kolter2006learning,Shafiq2009AFF,saxe2015deep}.
Nhiều tính năng tĩnh đã được đề xuất để trích xuất các tính năng từ tệp nhị phân: printable  strings \cite{schultz2001data}, import tables, opcodes, informational entropy, \cite{weber2002toolkit}, byte n-grams \cite{abou2004n}, two dimensional byte entropy histograms \cite{saxe2015deep}.
Nhiều tính năng khác cũng đã được đề xuất trong Microsoft Malware Classification Challenge trên Kaggle \cite{ronen2018microsoft}, như là opcode images, various decompiled assembly features, và các thống kê tổng hợp.
Ngoài ra, lấy cảm hứng từ sự thành công của các mô hình học tập sâu rộng đến đầu cuối trong xử lý hình ảnh và xử lý ngôn ngữ tự nhiên, Raff et al. giới thiệu phát hiện phần mềm độc hại từ raw byte sequences \cite{raff2017malware}.
Có lẽ các mô hình state-of-the-art sẽ thay đổi thành các mô hình học sâu end-to-end trong những tháng hoặc năm tiếp theo, nhưng các hand-crafted features có thể tiếp tục có liên quan do định dạng có cấu trúc của phần mềm độc hại.

Không gian tính năng có thể trở nên lớn, trong trường hợp đó các phương pháp như locality-sensitive hashing \cite{bayer2009scalable}, feature hashing \cite{jang2011bitshred} hoặc random projections \cite{fradkin2003experiments}đã được áp dụng trong phát hiện phần mềm độc hại.
Tuy nhiên, ngay cả sau khi áp dụng việc giảm kích thước, vẫn còn một số lượng lớn các tính năng, có thể gây ra các vấn đề về khả năng mở rộng cho một số thuật toán học máy. 
Neural networks đã nổi lên như một giải pháp thay thế có thể mở rộng do những tiến bộ đáng kể trong thuật toán đào tạo \cite{almasi2016review}.
Nhiều phương pháp sử dụng neural networks đã được giới thiệu \cite{saxe2015deep, dahl2013large, kephart1995biologically, benchea2014combining} mặc dù không có cách so sánh kết quả rõ ràng vì các tập dữ liệu khác nhau.

Một lựa chọn phổ biến khác là ensemble of trees, có thể mở rộng hợp lý một cách hợp lý bằng cách lấy mẫu không gian tính năng trong mỗi lần lặp \cite{breiman2001random}.
Cây quyết định có thể thích ứng tốt với các loại dữ liệu khác nhau và linh hoạt với nhiều tỷ lệ giá trị trong vectơ đặc trưng, vì vậy chúng mang lại hiệu suất tốt ngay cả khi không có tiêu chuẩn hóa dữ liệu.
